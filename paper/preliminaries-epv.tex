% !TEX root = ./main.tex

We now briefly discuss the mechanized reasoning tools we use for our 
proofs and give an overview of \mlkem and how it evolved from \kyber.
The cryptographic definitions and notation we use are fairly standard
and are given in~\cite{fullversion}.

\subsection{The EasyCrypt proof assistant}
\label{subsec:easycrypt}

EasyCrypt\footnote{\url{https://easycrypt.info}}~\cite{Barthe2014} is a proof
assistant for formalizing proofs of cryptographic properties.
%%
Its primary feature is the Probabilistic Relational Hoare Logic (pRHL), which
we use throughout to prove equivalences between games, sometimes conditioned by
the non-occurrence of a failure event. We call such conditional equivalence
steps \emph{up-to-bad} reasoning in later section.
%%
pRHL is designed to support reasoning about equivalences of probabilistic
programs while reasoning only locally (within oracles) and without reasoning
about the distribution of specific variables---essentially keeping track only
of the fact that variables in one program are distributed identically to
variables in the other, but not keeping track of what that distribution may be.
%%
This logic has proved highly expressive for the bulk of cryptographic proof
work. However, some steps require more global reasoning (about the entire
execution) or keeping track of the distribution of individual variables. 
%%
Logical rules to support such reasoning steps are implemented in EasyCrypt, but
are often unwieldy to apply in concrete context. The EasyCrypt team has, over
the years, developed a number of generic libraries that abstract those more
complex reasoning rules into ``game transformations'' or equivalence results
that can be instantiated as part of other proofs.

Our proof makes extensive use of three of those generic libraries, which we
outline below and discuss more in depth 
in~\cite{fullversion}:
\begin{description}
  \item[plug-and-pray] provides a formalized generic argument for bounding the
    security loss of reductions that guess an instance, session or query (among
    a bounded-sized set) will lead to a successful run;
  \item[hybrid] provides a formalized and generic argument for bounding the
    distance between two games that differ only in one oracle, but where the
    transition must be done query-by-query for the purpose of the proof;
  \item[PROM] provides a generic argument, initially intended to apply to
    programmable random oracles, that encapsulates the widely used argument
    that a ``value is random and independent of the adversary's view''.
\end{description}

\subsection{Background on \mlkem}
\label{sec:kyber_to_mlkem}

\subheading{From \kyber to \mlkem.}
The original NIST PQC round-1 submission of
\kyber described the scheme as a two-stage construction consisting of
a module-lattice instantiation of the IND-CPA-secure LPR encryption
scheme~\cite{EC:LyuPeiReg10,Eurocrypt2010PeikertSlides} and a
\emph{``slightly tweaked''} Fujisaki-Okamoto (FO) transform~\cite{C:FujOka99,JC:FujOka13} 
to build an IND-CCA-secure KEM. 
The reduction linking IND-CPA security of the encryption scheme to MLWE 
was not spelled out in the NIST submission document.  
For the proof of IND-CCA security of the full scheme the
submission mostly referred to previous work on the security of the FO
transform, most notably~\cite{TCC:HofHovKil17} and \cite{EPRINT:SaiXagYam17}.
Less than two months after the submission documents were made public,
D'Anvers pointed out\footnote{See \url{https://groups.google.com/a/list.nist.gov/g/pqc-forum/c/PX_Wd11BecI}.}
that there are major obstacles to
proving IND-CPA security from MLWE in \kyber.  The reason is that
unlike the original LPR scheme, round-1 \kyber included a
public-key-compression step, which invalidates an assumption in the
LPR proof. As a consequence, the \kyber submission team decided to
tweak the submission for round 2 of NIST PQC by removing the
public-key compression (and adjusting some other parameters to obtain
similar performance); some more tweaks to parameters were proposed for
the round-3 version of \kyber.

\begin{algorithm}[tb]
  \caption{$\kpke.\PKEGen()$: key generation}
  \label{alg:gen}
  \begin{algorithmic}[1]
    \Ensure Secret key $\sk \in \Rq^k$ and public key $\pk \in \hat{\Rq}^k \times \{0,1\}^{256}$
    \State $d \getsr \{0,1\}^{256}$
    \State $(\rho,\sigma) \gets \ROG(d)$
    \State $\hat{\Apo} \gets \Parse(\XOF(\rho))$
    \State $\spo, \epo \gets \CBD_{\eta_1}(\PRF_\sigma)$ \Comment{$\spo, \epo \in \Rq^k$}
    \State $\hat{\spo} \gets \NTT(\spo)$
    \State $\hat{\epo} \gets \NTT(\epo)$
    \State $\hat{\tpo} \gets \hat{\Apo} \pw \hat{\spo} + \hat{\epo}$
    \State \Return $\sk = \hat{\spo}$ and $\pk=(\hat{\tpo},\rho)$
  \end{algorithmic}
\end{algorithm}

\begin{algorithm}[tb]
  \caption{$\kpke.\PKEEnc(\pk, m)$: encryption}
  \label{alg:enc}
  \begin{algorithmic}[1]
    \Require Public key $\pk  = (\hat{\tpo}, \rho) \in \Rq^k \times \{0,1\}^{256}$, message $m \in \{0,1\}^{256}$
    \Ensure Ciphertext $c \in \R_{d_u}^k \times \R_{d_v}$
    \State $r \getsr \{0,1\}^{256}$
    \State $\hat{\Apo} \gets \Parse(\XOF(\rho))$
    \State $\rpo \gets \CBD_{\eta_1}(\PRF_r)$ \Comment{$\rpo \in \Rq^k$}
    \State $\epo_1, e_2 \gets \CBD_{\eta_2}(\PRF_r)$ \Comment{$\epo_1 \in \Rq^k, e_2 \in \Rq$}
    \State $\hat{\rpo} \gets \NTT(\rpo)$\label{line:kybercpa-enc:nttr}
    \State $\upo \gets \INVNTT(\hat{\Apo}^T \pw \hat{\rpo}) + \epo_1$
    \State $v \gets \INVNTT(\hat{\tpo}^T\pw\hat{\rpo}) + e_2 + \ToPoly(m)$ 
    \State $\cpo_1 \gets \KyberCompress(\upo, d_u)$
    \State $c_2 \gets \KyberCompress(v, d_v)$
    \State \Return $c = (\cpo_1, c_2)$
  \end{algorithmic}
\end{algorithm}

\begin{algorithm}[tb]
  \caption{$\kpke.\PKEDec(\sk,c)$: decryption}
  \label{alg:dec}
  \begin{algorithmic}[1]
    \Require Secret key $\sk = \hat{\spo} \in \Rq^k$ and ciphertext $c = (\cpo_1, c_2)\in \R_{d_u}^k \times \R_{d_v}$
    \Ensure Message $m \in \{0,1\}^{256}$
    \State $\tilde{\upo} \gets \KyberDecompress(\cpo_1, d_u)$
    \State $\tilde{v} \gets \KyberDecompress(c_2, d_v)$
    \State $m \gets \ToMsg(\tilde{v}-\INVNTT(\hat{\spo}^T\pw\NTT(\tilde{\upo})))$
    \State \Return $m$
\end{algorithmic}
\end{algorithm}

Through the further course of the NIST PQC project, issues were
identified also in the second part of \kyber's security argument, 
which establishes IND-CCA security through a ``tweaked'' FO transform.  
It turned out that existing QROM proofs of the FO transform,
e.g., from~\cite{TCC:HofHovKil17}, did not apply and that proving the particular
variant of the FO transform is not straightforward~\cite[Sec.~5.4]{EC:GruMarPat22}. 
Although \kyber's variant of the FO transform was eventually proven
secure~\cite{PKC:MarXag23,EPRINT:BarHul23}, NIST decided, 
after discussion on the pqc-forum mailing list, 
to revert to a more ``vanilla'' version of the FO
transform for \mlkem.

Our mechanized proof validates the final outcome of these evolutions, 
which led to the draft specification of \mlkem. 
In our models we do \emph{not} include
the public-key validation step at the beginning of encapsulation, 
which NIST introduced in the \mlkem draft standard. 
The reason is that, as far as we know, this is the last major 
open discussion topic for the final standard and it seems rather 
unlikely that it will be included in the final version of \mlkem.

\subheading{Technical description of \mlkem.}
We give high-level algorithmic descriptions of \kpke, 
the IND-CPA-secure public-key encryption scheme underlying \mlkem,
in \cref{alg:gen,alg:enc,alg:dec} and of the full IND-CCA-secure KEM
in \cref{alg:ccagen,alg:ccaenc,alg:ccadec}. For a more implementation-oriented description 
that operates on byte arrays, see \cite[Algs.~12--14]{fips203}.

\mlkem works in the ring $\Rq = \Zq[X]/(X^n + 1)$ with $q=3329$ and $n=256$.
The core operations are on small-dimension vectors and matrices over $\Rq$;
the dimension depends on the parameter $k$, which is different for different
parameter sets of \mlkem: 
\mlkem (NIST security level 1) uses $k=2$,
\mlkem (NIST security level 3) uses $k=3$, and
\mlkem (NIST security level 5) uses $k=4$.
We denote elements in $\Rq$ with regular lower-case letters (e.g., $v$);
vectors over $\Rq$ with bold-face lower-case letters (e.g., $\upo$),
and matrices over $\Rq$ with bold-face upper-case letters (e.g., $\Apo$).


\begin{algorithm}[tb]
  \caption{$\mlkem.\KEMGen()$: key generation}
  \label{alg:ccagen}
  \begin{algorithmic}[1]
    \Ensure Secret key $\sk = (\sk', \pk, \ROH(\pk), z) \in \Rq^k \times (\hat{\Rq}^k \times \{0,1\}^{256}) \times \{0,1\}^{256} \times \{0,1\}^{256}$ and public key $\pk \in \hat{\Rq}^k \times \{0,1\}^{256}$
    \State $(\pk,\sk') \gets \kpke.\PKEGen()$
    \State $z \gets \{0,1\}^{256}$
    \State $\sk \gets (\sk'\cat \pk\cat \ROH(\pk)\cat z)$
    \State \Return $(\sk, \pk)$
  \end{algorithmic}
\end{algorithm}


\begin{algorithm}[tb]
  \caption{$\mlkem.\KEMEnc(\pk)$: encapsulation}
  \label{alg:ccaenc}
  \begin{algorithmic}[1]
    \Require Public key $\pk \in \hat{\Rq}^k \times \{0,1\}^{256}$
    \Ensure Ciphertext $c \in \R_{d_u}^k \times \R_{d_v}$ and shared key $K \in \{0,1\}^{256}$
    \State $m\gets \{0,1\}^{256}$
    \State $(K,r) \gets \ROG(m\cat \ROH(\pk))$\Comment{$(K,r) \in \{0,1\}^{256} \times \{0,1\}^{256}$}
    \State $c \gets \kpke.\PKEEnc(\pk, m, r)$
    \State \Return $(c, K)$
  \end{algorithmic}
\end{algorithm}


\begin{algorithm}[tb]
  \caption{$\mlkem.\KEMDec(c,\sk)$: decapsulation}
  \label{alg:ccadec}
  \begin{algorithmic}[1]
    \Require Ciphertext $c \in \R_{d_u}^k \times \R_{d_v}$ and secret key $\sk = (\sk', \pk, \ROH(\pk), z) \in \Rq^k \times (\hat{\Rq}^k \times \{0,1\}^{256}) \times \{0,1\}^{256} \times \{0,1\}^{256}$
    \Ensure Shared key $K \in \{0,1\}^{256}$
    \State $m'\gets \kpke.\PKEDec(\sk',c)$
    \State $(K',r') \gets \ROG(m'\cat \ROH(\pk))$
    \State $c' \gets \kpke.\PKEEnc(\pk, m', r')$
    \State $K^- \gets \ROJ(z\cat c)$
    \If{$c = c'$} \textbf{then} $K \gets K'$ \textbf{else} $K \gets K^-$\Comment{Requires branch-free conditional}
    \EndIf 
    \State \Return $K$
  \end{algorithmic}
\end{algorithm}




In these descriptions, \XOF is an extendable output function that in 
\mlkem is instantiated with SHAKE-128~\cite{fips202}. 
\Parse interprets outputs of \XOF as sequence of $12$-bit unsigned integers 
and runs rejection sampling to obtain coefficients that look uniformly random modulo $q$. 
The function $\ROH$ is instantiated with SHA3-256, $\ROG$ is instantiated with SHA3-512
and $\ROJ$ is instantiated with SHAKE256 with output length fixed to 32 bytes.
$\CBD_\eta$ denotes sampling coefficients from a centered 
binomial distribution with parameter $\eta$;\footnote{This means we have $B(n,p)$ with $p=1/2$, 
$n=2\eta$ and expected value shifted to $0$.}
extension from coefficients to (vectors of) polynomials is done by sampling each
coefficient independently from $\CBD_\eta$. For example, \mlkemmid uses $\eta_1 = \eta_2 = 2$.
The sampling routine is parameterized by a pseudorandom function $\PRF_k$ with key $k$.
\NTT is the number-theoretic transform of a polynomial in $\Rq$. 
Both input and output of \NTT can be written as a sequence of $256$ coefficients
in $\Zq$ and typical implementations perform the transform inplace. 
However, output coefficients do not have any meaning as a polynomial in $\Rq$.
We therefore denote the output domain as $\hat{\Rq}$; we apply
the same notation for elements in $\hat{\Rq}$, e.g., $\hat{u} = \NTT(u)$.
Application of $\NTT$ to vectors and matrices over $\Rq$ is done element-wise.
\KyberCompress compresses elements in $\Rq$ (or $\Rq^k)$ by rounding coefficients 
to a smaller modulus $d_v$ (or $d_u$).
\KyberDecompress is an approximate inverse of \KyberCompress.
\ToPoly maps $256$-bit strings to elements in $\Rq$ by mapping a zero bit to
a zero coefficient and mapping a one bit to a $\frac{q}{2}$ coefficient;
\ToMsg rounds coefficients to bits to recover a message from a noisy version
of a polynomial generated by \ToPoly.


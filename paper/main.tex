\documentclass[conference,compsoc]{IEEEtran}

\usepackage[nocompress]{cite}
\usepackage[pdftex]{graphicx}
\usepackage{amsmath}
\usepackage{algorithmic}
\usepackage{array}
\usepackage[caption=false,font=footnotesize,labelfont=sf,textfont=sf]{subfig}
\usepackage{stfloats}
\usepackage{url}
\usepackage{todonotes}

\hyphenation{for-mo-sa}

\begin{document}

\title{Formally Verified Correctness Bounds for Lattice-Based Cryptography}

% % author names and affiliations
% % use a multiple column layout for up to three different
% % affiliations
\author{\IEEEauthorblockN{EC Frodo KEM Team}
  \IEEEauthorblockA{Formosa Crypto + CQT}}
% \and
% \IEEEauthorblockN{Homer Simpson}
% \IEEEauthorblockA{Twentieth Century Fox\\
% Springfield, USA\\
% Email: homer@thesimpsons.com}
% \and
% \IEEEauthorblockN{James Kirk\\ and Montgomery Scott}
% \IEEEauthorblockA{Starfleet Academy\\
% San Francisco, California 96678-2391\\
% Telephone: (800) 555--1212\\
% Fax: (888) 555--1212}}

% conference papers do not typically use \thanks and this command
% is locked out in conference mode. If really needed, such as for
% the acknowledgment of grants, issue a \IEEEoverridecommandlockouts
% after \documentclass

% for over three affiliations, or if they all won't fit within the width
% of the page (and note that there is less available width in this regard for
% compsoc conferences compared to traditional conferences), use this
% alternative format:
% 
%\author{\IEEEauthorblockN{Michael Shell\IEEEauthorrefmark{1},
%Homer Simpson\IEEEauthorrefmark{2},
%James Kirk\IEEEauthorrefmark{3}, 
%Montgomery Scott\IEEEauthorrefmark{3} and
%Eldon Tyrell\IEEEauthorrefmark{4}}
%\IEEEauthorblockA{\IEEEauthorrefmark{1}School of Electrical and Computer Engineering\\
%Georgia Institute of Technology,
%Atlanta, Georgia 30332--0250\\ Email: see http://www.michaelshell.org/contact.html}
%\IEEEauthorblockA{\IEEEauthorrefmark{2}Twentieth Century Fox, Springfield, USA\\
%Email: homer@thesimpsons.com}
%\IEEEauthorblockA{\IEEEauthorrefmark{3}Starfleet Academy, San Francisco, California 96678-2391\\
%Telephone: (800) 555--1212, Fax: (888) 555--1212}
%\IEEEauthorblockA{\IEEEauthorrefmark{4}Tyrell Inc., 123 Replicant Street, Los Angeles, California 90210--4321}}

% make the title area
\maketitle

\begin{abstract}
The abstract goes here.
\end{abstract}

\IEEEpeerreviewmaketitle



\section{Introduction}

Story
\begin{itemize}
\item Lattice-based crypto security and correctnes proofs have been formally verified
\item Correctness bounds are still estimated heuristically and/or using non-verified code
\item Mention the famous python script and tell its story
\item We give the first fully verified proof that bounds reported for PQ constructions, namely PQ KEM are correct
\item We develop an EasyCrypt extension that allows to compute formally verified upper bounds for the probabilities of events over the reals
\item We give a new proof of correctness and security for the PKE underlying Frodo-KEM and demonstrate our approach to give a concrete bound for the it's cryptographic correctness that confirms the bounds claimed in the NIST submission
\item We also confirm the (heuristic) bounds claimed by Kyber and Saber.
\end{itemize}

\section{Preliminaries}

\section{Correctness in Lattice-Based PKEs}

\section{Computing formally-verified upper-bounds in EasyCrypt}

\section{Frodo KEM PKE}

\section{Other lattice-based constructions}

\section{Conclusions and future work}

\bibliographystyle{IEEEtran}
\bibliography{local.bib}

\end{document}